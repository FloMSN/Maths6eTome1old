\begin{enigme}[La constante de Champernowne]


Soit $x$ et $y$ deux réels tels que $x<y$.\\
Définissons la suite  $(d_n)_{n\geqslant0}$ telle que $d_n=\dfrac{\lfloor10^n y\rfloor}{10^n}$ où $\lfloor a\rfloor$ désigne la partie entière de $a$.\\
\begin{enumerate}
\item À quel ensemble les nombres $d_n$  appartiennent-ils ? \hspace{1cm}
\begin{minipage}{0.5\linewidth}
\begin{colitemize}{5}
\item $\mathbb{N}$ ? \item $\mathbb{Z}$ ? \item $\mathbb{D}$ ? \item $\mathbb{Q}$ ? \item $\mathbb{R}$ ?
\end{colitemize}
\end{minipage}
\item
\begin{enumerate}
\item Montrer  que pour tout $n\in\mathbb{N}$, on a l'encadrement $\dfrac{10^ny-1}{10^n}< d_n\leqslant y$.
\item En déduire $\displaystyle\lim_{n\to+\infty}d_n$.
\end{enumerate}
\item \begin{enumerate}
\item Montrer que, quel que soit $\varepsilon>0$, il existe un entier naturel $N$ tel que pour tout $n\geqslant N$, $|d_n-y|< \varepsilon$.
\item En posant $\varepsilon=y-x$, en déduire que $x\leqslant d_N \leqslant y$.
\end{enumerate}
\end{enumerate}

\begin{center}
\begin{minipage}{1\linewidth}
\begin{cadre}[A1][A4]
On vient de montrer qu'entre deux réels, il existe toujours un décimal et donc toujours un rationnel.\\
On dit que l'ensemble des rationnels $\mathbb{Q}$ est \textbf{dense} dans l'ensemble des réels $\mathbb{R}$.
 \end{cadre}
\end{minipage}
\end{center}

La \textbf{fonction de Dirichlet} D et la \textbf{fonction de Thomae} T sont deux fonctions définies sur $\mathbb{R}$ par  :
\[\text{D}(x)=\left\{\begin{matrix} 1  &  \text{si~} x\in\mathbb{Q} \\ 0  & \text{si~} x\notin\mathbb{Q} \end{matrix}\right.
\text{~~et~~} \text{T}(x)=\left\{\begin{array}{@{}ll} 0  & \text{si~} x\notin\mathbb{Q} \\ 1 & \text{si~} x=0 \\ \dfrac{1}{q} & \text{si~} x=\dfrac{p}{q} \text{~est une fraction irréductible}\end{array}\right.\]
Introduite par Dirichlet\footnote{Johann Peter Gustav Lejeune Dirichlet (1805–1859), mathématicien allemand}  en 1829, la fonction D est discontinue partout ce que le résultat établi précédemment montre. Cette fonction est appelée aussi \textbf{fonction indicatrice des rationnels}.\\
Introduite par Thomae\footnote{Carl Johannes Thomae (1840–1921), mathématicien allemand} en 1875, la fonction T est continue en tout nombre irrationnel mais  discontinue en tout nombre rationnel. Cette fonction est appelée aussi la \textbf{fonction popcorn} (voir sa représentation ci-dessous !).
\begin{center}

\end{center}


\end{enigme} 



%\begin{enigne}[La constante de Champernowne]
%Ce nombre, inventé par le mathématicien anglais David Gawen Champernowne en 1933, commence par 0,123456789101112131415... .
%\begin{enumerate}
%\item Quelle est la particularité de cette constante ? Donne les dix décimales suivantes.
%\item Quelle est l'arrondi, au cent-milliardième près, de cette constante ?
%\end{enumerate}
%\end{enigme} 

%\begin{enigne}[Défis]
%\begin{enumerate}
%\item Combien de fois faudrait-il utiliser le chiffre 1 si l'on voulait écrire tous les nombres entiers de 1 à 999 ? Et le chiffre 9 ?
%Donne le nombre de mots utilisés pour écrire tous les entiers plus petits que 100.
%\end{enumerate}
%\end{enigme} 

%Calculatrices infernales (d'après Apmep)
%\begin{enigne} Sur la calculatrice d'Aïsha, la touche pour afficher la virgule ne fonctionne plus et la touche « $=$ » ne peut fonctionner qu'une seule fois par ligne de calcul. Comment peut‑elle trouver le résultat de $(17,32 \cdot 45,3) + 15,437$ ?
%\end{enigme} 

%\begin{enigne} Bruce vient de faire tomber sa calculatrice. Elle ne comporte plus que les chiffres, la virgule et les quatre opérations, mais quand on appuie sur « $+$ » elle ajoute 1, quand on appuie sur « $-$ » elle retranche 1, quand on appuie sur la touche « $\times$ » elle multiplie par 10 et quand on appuie sur la touche « $\div$ » elle divise par 10.
%\begin{enumerate}
%\item Romain emprunte la calculatrice de Bruce. Il tape 27,2 puis appuie ensuite sur les touches « $\times$ », « $\times$ », « $+$ », « $+$ », « $-$ », « $\div$ », « $\div$ », « $\div$ », « $+$ », « $\times$ ». Quel résultat Romain trouve-t-il ?

%\item Comment peut‑il passer en sept opérations :
%\begin{itemize}
%\item de 3,14 à 300 ?
%\item de 3,14 à 297 ?
%\item de 297 à 0,2 ?
%\end{itemize}
%\item Tu viens de passer de 3,14 à 0,2 en quatorze opérations. Trouve un chemin qui permette de faire cela avec le minimum d'opérations. Compare avec tes camarades.

%Trouve un chemin qui permette de passer de 5 à 4,99 en un minimum d'opérations puis compare avec tes camarades.
%\end{enumerate}
%\end{enigme} 
