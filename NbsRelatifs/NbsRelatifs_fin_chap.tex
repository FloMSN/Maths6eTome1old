\begin{enigme}[Puce « olympique »]

Lorsqu'elle utilise sa patte gauche seule, elle fait des bonds de 6 cm.

Lorsqu'elle utilise sa patte droite seule, elle fait des bonds de 4 cm.

Et lorsqu'elle saute « à pattes jointes », elle fait des bonds de 34 cm !

Quel est le nombre \underline{minimum} de bonds qu'elle doit réaliser pour parcourir exactement 20 m ?

Même question avec 35 m.
 
 \end{enigme}
 
 \vspace*{2em}
 
%%%%%%%%%%%%%%%%%%%%%%%%%%%%%%%%%%%%%%%%%%%%%%%%%%%%%%%%%%%%%%%%%%%%%

\begin{enigme}[Abracadabra]

Un magicien donne la formule magique à son apprenti.

« Voici la formule magique, elle est formée d'une infinité de séquences $AB$ et $BA$. Lorsque tu l'auras recopiée, tu seras mon égal ».

L'apprenti, pour gagner du temps, remplace chaque bloc $AB$ par la lettre $A$ et chaque bloc $BA$ par la lettre $B$, et, oh stupeur ! La formule magique reste inchangée !

Quelles sont les 2002\up{ème}, 2003\up{ème}, 2004\up{ème}, 2005\up{ème}, 2006\up{ème}, 2007\up{ème} et 2008\up{ème} lettres de la formule magique ?

\end{enigme} 
