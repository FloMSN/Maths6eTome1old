\begin{exercice}[Histoire]
Classe les dates des événements suivants par ordre chronologique :
\begin{itemize}
 \item Signature du pacte fédéral d'alliance perpétuelle entre les communautés d'Uri, Schwytz et Nidwald : $\numprint{1291}$ ;
 \item La mort de Toutankhamon : $\numprint{-1327}$ ;
 \item L'éruption du Vésuve qui ensevelit Pompéi sous les cendres : $79$ ;
 \item La défaite d'Alésia : $52$ av. J.-C. ;
 \item La mort de Léonard de Vinci : $\numprint{1519}$ ;
 \item La naissance de Jules César : $100$ av. J.-C. ;
 \item Le début de la guerre de $100$ ans : $\numprint{1337}$ ;
 \item La naissance de Socrate : $470$ av. J.-C. ;
 \item Ta date de naissance.
 \end{itemize}
\end{exercice}


\begin{exercice}[Géographie]
Complète le tableau en recherchant les altitudes maximales et les profondeurs (altitudes minimales) :\\[0.5em]
\begin{cltableau}{\linewidth}{3}
 \hline
 & Sommets, altitude maximale & Profondeurs, altitude minimale \\\hline
 Afrique & Kilimandjaro & Lac Assal\\
 & & \\\hline
 Europe & Elbrouz & Mer Caspienne \\
 & & \\\hline
 Amérique du sud & Aconcagua & Rio Negro \\\hline
 Asie & Everest & Mer Morte \\
 & & \\\hline
 Océan pacifique & Mauna Kea & La fosse des Mariannes \\
 & & \\\hline
 \end{cltableau}
  \vspace{0.3cm}
\begin{enumerate}
 \item Quel est le sommet le plus haut ?
 \item La mer Caspienne est-elle plus profonde que le Rio Negro ?
 \item En Afrique, combien de mètres séparent le point le plus profond du lac Assal et le sommet du Kilimandjaro ?
 \item Un poisson se trouve tout au fond de la mer morte. À quelle distance se trouve-t-il de la surface de l’eau ?
 \end{enumerate}
\end{exercice}


\begin{exercice}[Repères]
Dans chaque cas, trace un repère en choisissant judicieusement l'unité pour pouvoir placer tous les points :
\begin{enumerate}
 \item $A(-3 ; 3)$ ; $B(1 ; 4)$ et $C(5 ; 2)$ ; 
 \item $D(-13 ; 8)$ ; $E(25 ; 14)$ et $F(-35 ; 22)$ ;
 \item $G(-83 ; -8)$ ; $H(72 ; -55)$ et $I(-15 ; 32)$.
 \end{enumerate}
\end{exercice}


\begin{exercice}[Coordonnées mystères]
\begin{enumerate}
 \item Construis un repère et places-y les points $A$, $B$, $C$, $D$, $E$ et $F$ sachant que :
 \begin{itemize}
  \item Les valeurs des coordonnées des six points sont :
$0$ ; $0$ ; $3$ ; $4$ ; $-2$ ; $2$ ; $-4$ ; $1$ ; $-1$ ; $3$ ; $-1$ et $-2$ ;
  \item Les ordonnées des six points sont toutes différentes et si on range les points dans l'ordre décroissant de leurs ordonnées, on obtient : $E$, $B$, $F$, $C$, $A$ et $D$ ;
  \item Les abscisses de tous les points sauf $D$ sont différentes et si on range les points dans l'ordre croissant de leurs abscisses, on obtient : $F$, $B$, $A$, $E$ et $C$ ;
  \item Le point $E$ est sur l'axe des ordonnées ;
  \item L'ordonnée de $E$ est l'opposé de l'abscisse de $F$ ;
  \item Le point $C$ est sur l'axe des abscisses à une distance de 3 de l'origine ;
  \item Les deux coordonnées du point $B$ sont opposées.
  \end{itemize}
 \item Que dire de la droite $(CD)$ ? Justifie ta réponse.
 \end{enumerate}
\end{exercice}



