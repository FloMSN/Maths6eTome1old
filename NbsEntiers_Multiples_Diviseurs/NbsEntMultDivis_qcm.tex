\begin{acquis} % enlever peut-être le lien internet
\begin{itemize}
\item trouver des diviseurs d'un nombre (et tous les diviseurs si ce nombre n'est pas trop grand);
\item trouver des multiples d'un nombre; 
\item décomposer un nombre en produits de facteurs premiers;
\item calculer avec des puissances;
\item calculer le PGCD de deux nombres entiers;
\item résoudre des problèmes nécessitant l'utilisation du PGCD;
\item calculer le PPMC de deux nombres entiers.
\end{itemize}
\end{acquis}

\QCMautoevaluation{Pour chaque question, plusieurs réponses sont
  proposées.  Déterminer celles qui sont correctes.}

\begin{QCM}
  \begin{GroupeQCM} 
    \begin{exercice}
      84 est divisible par \ldots
      \begin{ChoixQCM}{4}
      \item 5
      \item 9
      \item 2
      \item3
      \end{ChoixQCM}
\begin{corrige}
     \reponseQCM{cd} 
   \end{corrige}
    \end{exercice}

    \begin{exercice}
      150 est divisible par \ldots
      \begin{ChoixQCM}{4}
      \item 3
      \item 2
      \item 5
      \item 10
      \end{ChoixQCM}
\begin{corrige}
     \reponseQCM{abcd}
   \end{corrige}
    \end{exercice}
    
    \begin{exercice}
      435 est \ldots
      \begin{ChoixQCM}{4}
      \item Un multiple de 5
      \item Un diviseur de 5
      \item Divisible par 5
      \item Un multiple de 3
      \end{ChoixQCM}
\begin{corrige}
     \reponseQCM{acd}
   \end{corrige}
    \end{exercice}

    \begin{exercice}
      17 est \ldots
      \begin{ChoixQCM}{4}
      \item Un diviseur de 3\,672
      \item Un multiple de 17
      \item Le seul diviseur de 17
      \item Un multiple de 8,5
      \end{ChoixQCM}
\begin{corrige}
     \reponseQCM{abd}
   \end{corrige}
    \end{exercice}

    \begin{exercice}
      Retrouve la (ou les) affirmation(s) vraie(s) :
      \begin{ChoixQCM}{4}
      \item Tout nombre entier est un multiple de 0
      \item Il existe toujours au moins un diviseur commun à deux entiers
      \item La liste des diviseurs d'un entier est infinie
      \item Un nombre entier est toujours divisible par lui‑même
      \end{ChoixQCM}
\begin{corrige}
     \reponseQCM{bd}
   \end{corrige}
    \end{exercice}
    
    \begin{exercice}
      15 est \ldots
      \begin{ChoixQCM}{4}
      \item Un diviseur commun à 30 et 45
      \item Le PGDC de 30 et 45
      \item Le plus grand multiple commun à 3 et 5
      \item Le plus grand des diviseurs communs à 60 et 135
      \end{ChoixQCM}
\begin{corrige}
     \reponseQCM{abd}
   \end{corrige}
    \end{exercice}
    
    \begin{exercice}
      Le PGCD de 12 et 18 est \ldots
      \begin{ChoixQCM}{4}
      \item 1
      \item 6
      \item 2
      \item 0
      \end{ChoixQCM}
\begin{corrige}
     \reponseQCM{b}
   \end{corrige}
    \end{exercice}

    \begin{exercice}
      24 est \ldots
      \begin{ChoixQCM}{4}
      \item Le PPMC de 6 et 4
      \item Un multiple commun à 8 et 6 
      \item Le PPMC de 8 et 6
      \item Un multiple de 48
      \end{ChoixQCM}
\begin{corrige}
     \reponseQCM{bc}
   \end{corrige}
    \end{exercice}
   
\end{GroupeQCM}
\end{QCM}

     



\begin{QCM}
  \begin{GroupeQCM} 

    \begin{exercice}
      $5^3 =$ \ldots
      \begin{ChoixQCM}{4}
      \item 15
      \item 8
      \item 125
      \item $03:05:00$
      \end{ChoixQCM}
\begin{corrige}
     \reponseQCM{c}
   \end{corrige}
    \end{exercice}
    
    \begin{exercice}
      51 est \ldots
      \begin{ChoixQCM}{4}
      \item Un nombre premier
      \item Un multiple de 7
      \item Divisible par 17
      \item Un diviseur de 102
      \end{ChoixQCM}
\begin{corrige}
     \reponseQCM{cd}
   \end{corrige}
    \end{exercice}
    
    \begin{exercice}
      Dans $4^3$, 3 est \ldots
      \begin{ChoixQCM}{4}
      \item La base
      \item L'exposant
      \item La puissance
      \item Le facteur
      \end{ChoixQCM}
\begin{corrige}
     \reponseQCM{bc}
   \end{corrige}
    \end{exercice}
    
    \begin{exercice}
      La décomposition en produits de facteurs premiers de 84 possède \ldots
      \begin{ChoixQCM}{4}
      \item 3 facteurs distincts
      \item Le facteur $2^3$
      \item 4 facteurs
      \item Deux facteurs premiers
      \end{ChoixQCM}
\begin{corrige}
     \reponseQCM{ac}
   \end{corrige}
    \end{exercice}
    
\end{GroupeQCM}
\end{QCM}

  
