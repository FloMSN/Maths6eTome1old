\begin{acquis}
\begin{itemize}
\item BlaBla1
\item BlaBla2
\item BlaBla3
\item BlaBla4
\item BlaBla5
\item BlaBla6
\end{itemize}
\end{acquis}

\QCMautoevaluation{Pour chaque question, plusieurs réponses sont
  proposées.  Déterminer celles qui sont correctes.} % Est-ce que c'est toujours le cas ?
% Enlever le link internet ...
\begin{QCM}
  \begin{GroupeQCM} % Est-ce qu'on les séparent en groupe ?
    \begin{exercice}
      Le produit est \ldots
      \begin{ChoixQCM}{4}
      \item le résultat d'une addition
      \item le résultat d'une soustraction
      \item le résultat d'une multiplication
      \item le résultat d'une division
      \end{ChoixQCM}
\begin{corrige}
     \reponseQCM{a} % J'ai mis comme réponse "a" partout !
   \end{corrige}
    \end{exercice}

\begin{exercice}
      Quelle est l'opération prioritaire dans le calcul de $(10 + 4) : 2 - 3 \cdot 2$ ?
      \begin{ChoixQCM}{4}
      \item l'addition
      \item la multiplication
      \item la division
      \item la soustraction
      \end{ChoixQCM}
\begin{corrige}
     \reponseQCM{a}
   \end{corrige}
    \end{exercice}
    
\begin{exercice}
      $20 - 4 + 3 \cdot 2 =$ \ldots
      \begin{ChoixQCM}{4}
      \item 38
      \item 6
      \item 22
      \item 10
      \end{ChoixQCM}
\begin{corrige}
     \reponseQCM{a}
   \end{corrige}
    \end{exercice}
    
\begin{exercice}
      La somme de 5 et du quotient de 10 par $2 =$ \ldots
      \begin{ChoixQCM}{4}
      \item 10
      \item 7,5
      \item 17
      \item 2,4
      \end{ChoixQCM}
\begin{corrige}
     \reponseQCM{a}
   \end{corrige}
    \end{exercice}
    
\begin{exercice}
      $\dfrac{12 - 2 \cdot 3}{3 \cdot 3 + 1} =$ \ldots
      \begin{ChoixQCM}{4}
      \item 3
      \item 0,6
      \item 0,5
      \item 2,5
      \end{ChoixQCM}
\begin{corrige}
     \reponseQCM{a}
   \end{corrige}
    \end{exercice}
    
\begin{exercice}
      Les termes sont \ldots
      \begin{ChoixQCM}{4}
      \item des nombres que l'on additionne
      \item des nombres que l'on soustrait
      \item des nombres que l'on multiplie
      \item des nombres que l'on divise
      \end{ChoixQCM}
\begin{corrige}
     \reponseQCM{a}
   \end{corrige}
    \end{exercice}
    
\begin{exercice}
      Les facteurs sont \ldots
      \begin{ChoixQCM}{4}
      \item des nombres que l'on additionne
      \item des nombres que l'on soustrait
      \item des nombres que l'on multiplie
      \item des nombres que l'on divise
      \end{ChoixQCM}
\begin{corrige}
     \reponseQCM{a}
   \end{corrige}
    \end{exercice}
    
\begin{exercice}
      Le calcul qui fait d'abord  l'addition, puis la division et enfin la soustraction est \ldots
      \begin{ChoixQCM}{4}
      \item $46 - (5 + 6 : 2)$
      \item $(46 - 5 + 6) : 2$
      \item $46 - (5 + 6) : 2$
      \item $46 - 5 + 6 : 2$
      \end{ChoixQCM}
\begin{corrige}
     \reponseQCM{a}
   \end{corrige}
    \end{exercice}
    
\begin{exercice}
      Si dans un calcul il n'y a pas de parenthèses prioritaires alors on effectue les additions et les soustractions avant les multiplications et les divisions.
      \begin{ChoixQCM}{2}
      \item VRAI
      \item FAUX
      \end{ChoixQCM}
\begin{corrige}
     \reponseQCM{a}
   \end{corrige}
    \end{exercice}
    
\begin{exercice}
      Si dans un calcul il n'y a que des additions et des soustractions, alors les additions sont prioritaires.
      \begin{ChoixQCM}{2}
      \item VRAI
      \item FAUX
      \end{ChoixQCM}
\begin{corrige}
     \reponseQCM{a}
   \end{corrige}
    \end{exercice}
    
\begin{exercice}
      Si dans un calcul on ne peut pas définir de priorité alors on effectue les calculs de gauche à droite.
      \begin{ChoixQCM}{2}
      \item VRAI
      \item FAUX
      \end{ChoixQCM}
\begin{corrige}
     \reponseQCM{a}
   \end{corrige}
    \end{exercice}

\end{GroupeQCM}
\end{QCM}

  