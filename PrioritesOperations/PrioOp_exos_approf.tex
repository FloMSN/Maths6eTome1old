\begin{exercice}[Recherche sur internet]
 \begin{enumerate}
  \item Essaie de trouver sur Internet à quelle date est apparue la première calculatrice ressemblant à celles qu'on utilise de nos jours.
  \item Avant l'apparition des « machines à calculer », comment effectuait-on les calculs ? Essaie de trouver plusieurs « ancêtres » de nos calculatrices modernes.
  \end{enumerate}
\end{exercice}


\begin{exercice}[Avec des mots]
\begin{center} $(4 + 3) \cdot (11 - 5)$ \end{center}
se lit de la façon suivante : « Le produit de la somme de 4 et 3 par la différence de 11 et 5. ». \\[0.75em]
Construis cinq phrases différentes en utilisant les mots et les nombres de la phrase ci-dessus et traduis chacune d’elle par un calcul.
\end{exercice}


\begin{exercice}[Question de bon sens]
Recopie et complète entre les chiffres par les signes  \textcolor{B1}{$+$} , \textcolor{B1}{$-$} , \textcolor{B1}{$\cdot$} , \textcolor{B1}{$:$} , \textcolor{B1}{$( et )$} pour que les égalités soient vraies.
\begin{colenumerate}{3}
 \item $3 3 3 3 = 0$
 \item $3 3 3 3 = 1$
 \item $3 3 3 3 = 2$
 \item $3 3 3 3 = 3$
 \item $3 3 3 3 = 4$
 \item $3 3 3 3 = 5$
 \item $3 3 3 3 = 6$
 \item $3 3 3 3 = 7$
 \end{colenumerate}
\end{exercice}


\begin{exercice}[Histoires de parenthèses]
Recopie les calculs suivants et place des parenthèses qui permettent de trouver le résultat indiqué :
\begin{enumerate}
 \item $9 - 3 \cdot 2 = 12$
 \item $16 + 8 : 4 - 3 = 24$
 \item $35 + 5 : 5 - 3 + 2 \cdot 4 = 13$
 \item $35 + 5 : 5 - 3 + 2 \cdot 4 = 4$
 \item $9 - 3 \cdot 2 = 3$
 \item $16 + 8 : 4 - 3 = 3$
 \item $35 + 5 : 5 - 3 + 2 \cdot 4 = 41$
 \item $35 + 5 : 5 - 3 + 2 \cdot 4 = 28$
 \item $35 + 5 : 5 - 3 + 2 \cdot 4 = 16$
 \end{enumerate}
\end{exercice}


\begin{exercice}[Le compte est bon]
Exemple : on doit obtenir 271 en utilisant les nombres 2, 5, 7, 8, 9 et 10. \\[0.75em]
On peut additionner, soustraire, multiplier ou diviser, mais il n’est pas permis d’utiliser le même nombre plusieurs fois. Par contre, il est permis de ne pas utiliser tous les nombres donnés. Une fois que tu as trouvé, effectue en détail.
\begin{center} Solution : $271 = (8 \cdot 2 + 5 + 7) \cdot 10 - 9$ \end{center}
\begin{enumerate}
 \item À l'aide des nombres 1, 2, 4, 5, 6 et 100, trouve 709 ;
 \item À l'aide des nombres 5, 6, 7, 8, 9 et 10, trouve 339 ;
 \item À l'aide des nombres 1, 3, 4, 5, 8 et 75, trouve 704 ;
 \item À l'aide des nombres 1, 2, 4, 9, 10 et 50, trouve 327 ;
 \item À l'aide des nombres 3, 4, 7, 8, 10 et 75, trouve 924 ;
 \item À l'aide des nombres 2, 2, 5, 5, 7 et 100, trouve 917.
 \end{enumerate}
\end{exercice}