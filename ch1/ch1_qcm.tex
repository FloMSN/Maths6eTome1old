\begin{acquis}
\begin{itemize}
\item BlaBla1
\item BlaBla2
\item BlaBla3
\item BlaBla4
\item BlaBla5
\item BlaBla6
\end{itemize}
\end{acquis}

\QCMautoevaluation{Pour chaque question, plusieurs réponses sont
  proposées.  Déterminer celles qui sont correctes.} % Est-ce que c'est toujours le cas ?

\begin{QCM}
  \begin{GroupeQCM} % Est-ce qu'on les séparent en groupe ?
    \begin{exercice}
      Dix-huit millions huit cents s'écrit :
      \begin{ChoixQCM}{4}
      \item 18\,800\,000
      \item 18\,000\,800
      \item 18\,800
      \item 18\,008\,100
      \end{ChoixQCM}
\begin{corrige}
     \reponseQCM{c}
   \end{corrige}
    \end{exercice}

    \begin{exercice}
      45 centaines est égal à :
      \begin{ChoixQCM}{4}
      \item 5 unités
      \item 450 dizaines
      \item 4 dizaines
      \item 45\,100
      \end{ChoixQCM}
      \begin{corrige}
     \reponseQCM{d}
   \end{corrige}
    \end{exercice}

    
    \begin{exercice}
      Un centième est :
      \begin{ChoixQCM}{4}
      \item plus grand qu'un dixième % espacement moche...
      \item égal à dix millièmes
      \item plus petit qu'un millième
      \item égal à dix  dixièmes
      \end{ChoixQCM}
      \begin{corrige}
     \reponseQCM{b}
   \end{corrige}
    \end{exercice}


    \begin{exercice}
      Une écriture décimale de 456 centièmes est :
      \begin{ChoixQCM}{4}
      \item 456,100
      \item 456\,100
      \item 4,56
      \item 4\,560 millièmes
      \end{ChoixQCM}
      \begin{corrige}
     \reponseQCM{ac}
   \end{corrige}
    \end{exercice}


    \begin{exercice}
      Le nombre $5 + 0,4 + 0,007$ peut aussi s'écrire :
      \begin{ChoixQCM}{4}
      \item 547 millièmes
      \item 5,47
      \item 5,407
      \item 5\,047 millièmes
      \end{ChoixQCM}
      \begin{corrige}
     \reponseQCM{abc}
   \end{corrige}
    \end{exercice}
    
        \begin{exercice}
      7 unités, 8 centièmes et 5 millièmes s'écrit :
      \begin{ChoixQCM}{4}
      \item 7,85
      \item 7,085
      \item 7,800\,500\,0
      \item 7,085\,0
      \end{ChoixQCM}
      \begin{corrige}
     \reponseQCM{abc}
   \end{corrige}
    \end{exercice}

        \begin{exercice}
      7 unités, 8 centièmes et 5 millièmes s'écrit :
      \begin{ChoixQCM}{4}
      \item 7,85
      \item 7,085
      \item 7,800\,500\,0
      \item 7,085\,0
      \end{ChoixQCM}
      \begin{corrige}
     \reponseQCM{abc}
   \end{corrige}
    \end{exercice}

    \begin{exercice}
    Soit ci-dessous la courbe représentative d'une fonction $f$.
\begin{center}
\psset{xunit=1.cm,yunit=1.cm,algebraic=true}
\begin{pspicture*}(-5.25,-1.2)(5.25,1.4)
\psgrid[subgriddiv=1,linewidth=0.5pt,gridcolor=A3,subgridcolor=A3,gridlabels=0pt](-6,-2)(6,3)
\psaxes[linewidth=0.8pt,Dx=1,Dy=1,ticksize=-2pt]{->}(0,0)(-5.25,-1.2)(5.25,1.4)
%\psaxes[linewidth=0.5pt,Dx=10,Dy=10]{->}(0,0)(1,1)
  \rput[0](1.9,2.5){\textcolor{B2}{$\mathscr{C}$}}
\psplot[linecolor=B2,plotpoints=5000,linewidth=0.8pt]{-5.25}{-1.05}{((x^2+x)/(x^2-1)-(x^2-x)/(-x^2+1))/8}
\psplot[linecolor=B2,plotpoints=5000,linewidth=0.8pt]{-0.95}{0}{((x^2+x)/(x^2-1)-(x^2-x)/(-x^2+1))/8}
\psplot[linecolor=B2,plotpoints=5000,linewidth=0.8pt]{0}{0.95}{((x^2+x)/(x^2-1)-(x^2-x)/(-x^2+1))/8+0.13}
\psplot[linecolor=B2,plotpoints=5000,linewidth=0.8pt]{2.05}{5.25}{(((x-1)^2+x-1)/((x-1)^2-1)-((x-1)^2-x+1)/(-(x-1)^2+1))/8}
\psline[linecolor=B2,linewidth=0.8pt](1.2,-1.2)(1.8,1.4)
\rput(0,0){\textcolor{Blanc}{$\bullet$}} \rput(0,0){\textcolor{B2}{$\circ$}}
\rput(0,0.13){\textcolor{B2}{$\bullet$}}
\end{pspicture*}
\end{center}
Il est certain que la fonction $f$ n'est pas continue :
      \begin{ChoixQCM}{4}
      \item en $-1$
      \item en $0$
      \item en $2$
      \item en $6$
      \end{ChoixQCM}
      \begin{corrige}
     \reponseQCM{ab}
   \end{corrige}
    \end{exercice}




\end{GroupeQCM}
\end{QCM}

  